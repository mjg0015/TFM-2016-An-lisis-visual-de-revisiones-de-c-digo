\capitulo{2}{Objetivos del proyecto}

Este apartado se ha desglosado en dos secciones. En la primera se enumeran los objetivos de carácter general asociados a los requisitos del proyecto. En la segunda se describen aquellos objetivos técnicos relacionados con las tecnologías y metodologías a utilizar.

\section{Objetivos generales}

Se desea desarrollar una herramienta con dos funcionalidades claramente diferenciadas:

\begin{itemize}
	\item Obtener y almacenar datos sobre revisiones de código realizadas en repositorios alojados en la plataforma GitHub.
	\item Representar gráficamente los datos en bruto a través de gráficas de diferentes tipos.
\end{itemize}

Con el desarrollo de una herramienta de este tipo se persigue lo siguiente:

\begin{itemize}
	\item Obtención de conocimiento e información útil sobre revisiones de código y revisores mediante el análisis visual de los datos.
	\item Creación de un almacén de datos cuyo contenido pueda ser utilizado en procesos de minería de datos en ámbitos de investigación.
	\item Contribuir a una mejora de las revisiones de código haciendo de éste un proceso más importante y útil para el desarrollo de software de calidad.
\end{itemize}


\section{Objetivos técnicos}

A nivel técnico, con la realización de este proyecto se busca aplicar los conocimientos obtenidos a lo largo del grado y máster, así como el aprendizaje de nuevas tecnologías y metodologías utilizadas en la actualidad:

\begin{itemize}
	\item Uso de la metodología ágil Scrum.
	\item Profundizar en la utilización de Git como sistema de control de versiones, y de forma concreta, la herramienta GitHub y sus funcionalidades particulares.
	\item Aprendizaje y uso de integraciones de GitHub como ZenHub para la gestión de proyectos.
	\item Aprendizaje y uso de técnicas de integración continua con herramientas como Travis, Heroku o Codebeat.
	\item Aprendizaje y uso de TypeScript (JavaScript tipado) junto con node.js.
	\item Aprendizaje y uso de jQuery.
	\item Utilización de bases de datos no relacionales (NoSQL), concretamente MongoDB.
	\item Creación de una aplicación distribuida cuyos elementos sean independientes formada por:
	
	\begin{itemize}
		\item API REST en el lado del servidor.
		\item Cliente de tipo SPA (\textit{Single Page Application}).
	\end{itemize}
\end{itemize}
