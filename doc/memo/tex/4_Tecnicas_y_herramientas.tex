\capitulo{4}{Técnicas y herramientas}

En este apartado se definen las diferentes técnicas y herramientas utilizadas a lo largo del desarrollo del trabajo.

\section{Metodologías ágiles}

Frente a las metodologías tradicionales, se ha optado por el uso de metodologías ágiles, las cuales priman \cite{agilealliance:agilemanifesto}:

\begin{itemize}
	\item Individuos e interacciones sobre procesos y herramientas.
	\item Software funcional sobre  documentación.
	\item Colaboración del cliente sobre negociación de contratos.
	\item Respuesta ante cambios sobre seguimiento de un plan.
\end{itemize}

\subsection{Scrum}

Concretamente se ha optado por el uso de Scrum como metodología ágil.

Scrum propone seguir un proceso de desarrollo iterativo e incremental a través de iteraciones denominadas sprints y de revisiones. Cada iteración debe finalizar con la entrega de una parte funcional del producto \cite{scrummanager}.

\section{Git}

Git es un sistema de control de versiones distribuido diseñado para manejar proyectos de cualquier tamaño con velocidad y eficiencia.

\begin{itemize}
	\item \url{https://www.git-scm.com}
\end{itemize}

\section{Github}

Github es una plataforma que permite alojar proyectos en repositorios de código que utilizan el sistema de control de versiones Git. También cuenta con funcionalidades para realizar revisiones de código. En este trabajo se va a utilizar esta herramienta como repositorio de código y como fuente de datos sobre revisiones de código en diferentes repositorios.

\begin{itemize}
	\item \url{https://www.github.com}
\end{itemize}

\subsection{Alternativas estudiadas}

Bitbucket es una alternativa a Github como herramienta de repositorio de código.

\begin{itemize}
	\item \url{https://bitbucket.org}
\end{itemize}

En el ámbito de herramienta de revisión de código se estudiaron Gerrit Code Review y Review Board.

\begin{itemize}
	\item Gerrit Code Review: \url{https://www.gerritcodereview.com/}
	\item Review Board: \url{https://www.reviewboard.org/}
\end{itemize}

\section{Node.js}

Node.js es un entorno de ejecución para JavaScript construido con el motor de JavaScript v8 de Chrome. En este trabajo se utiliza para ejecutar código JavaScript en el lado del servidor.

\begin{itemize}
	\item \url{https://nodejs.org}
\end{itemize}

\subsection{npm}

Node.js cuenta con npm, un gestor de paquetes con un amplio número de librerías registradas.

\begin{itemize}
	\item \url{https://www.npmjs.com/}
\end{itemize}

\section{TypeScript}

TypeScript es un superconjunto tipado de JavaScript que es compilado a JavaScript plano. Está desarrollado por Microsoft, y su uso puede mejorar la legibilidad y el entendimiento del código con respecto a JavaScript.

\begin{itemize}
	\item \url{https://www.typescriptlang.org/}
\end{itemize}

\subsection{TypeDoc}

TypeDoc es una herramienta para la generación de documentación para proyectos desarrollados en TypeScript. Su sintaxis es similar a la de JSDoc, pero simplificada ya que es capaz de leer e incluir los tipos de los parámetros definidos en el código.

\begin{itemize}
	\item \url{http://typedoc.org/}
\end{itemize}

\section{jQuery}

jQuery es una librería de JavaScript, rápida, versátil, liviana y llena de características. Simplifica la manipulación de documentos HTML, el tratamiento de eventos, uso de Ajax, etc. Es compatible con multitud de navegadores.

\begin{itemize}
	\item \url{https://jquery.com/}
\end{itemize}

\section{Sammy.js}

Sammy.js es un pequeño framework JavaScript que simplifica el desarrollo de aplicaciones web SPA.
En el proyecto se utiliza para realizar el enrutamiento en el cliente.

\begin{itemize}
	\item \url{http://sammyjs.org/}
\end{itemize}

\subsection{Alternativas estudiadas}

Como alternativas a Sammy se han estudiado AngularJS y React, dos conocidos frameworks para el desarrollo de aplicaciones web en el lado del cliente. Se ha escogido Sammy por su simplicidad frente a estos frameworks.

\begin{itemize}
	\item AngularJS: \url{https://angularjs.org/}
	\item React: \url{https://facebook.github.io/react/}
\end{itemize}

\section{Semantic-UI}

Semantic-UI es un framework para maquetación web que hace uso de una sintaxis similar al lenguaje natural que ayuda a crear un HTML más intuitivo.

\begin{itemize}
	\item \url{https://semantic-ui.com/}
\end{itemize}

\subsection{Alternativas estudiadas}

Como alternativa a Semantic-UI se ha estudiado Bootstrap. Elegimos el primero por que la versión actual de Bootstrap (3) fue liberada en 2013, y la nueva versión (4) aun se encuentra en fase de desarrollo.

\begin{itemize}
	\item \url{http://getbootstrap.com/}
\end{itemize}

\section{D3.js + C3.js}

D3.js es una librería JavaScript para la manipulación de documentos basados en datos. Permite representar datos de diversas maneras mediante el uso de HTML, SVG y CSS.

C3.js es una librería que contiene una serie de gráficos pre-diseñados desarrollados en D3.js, y gracias a ella se representan los diferentes gráficos de la parte del cliente del proyecto.
 
\begin{itemize}
	\item D3.js: \url{https://d3js.org/}
	\item C3.js: \url{http://c3js.org/}
\end{itemize}

\section{Visual Studio Code}

Visual Studio Code es un entorno de desarrollo integrado (IDE) multiplataforma desarrollado por Microsoft, cuenta con una herramienta integrada para el uso de Git, así como un elevado número de extensiones que permiten personalizar el editor para las necesidades concretas de cada proyecto.

\begin{itemize}
	\item \url{https://code.visualstudio.com/}
\end{itemize}

\section{Google Drive}

Google Drive es un servicio de almacenamiento en la nube, ofrece 15 GB de almacenamiento gratuito y se utiliza en este trabajo como sistema de copias de seguridad y como medio para mantener el entorno de desarrollo sincronizado en cualquier máquina.

\begin{itemize}
	\item \url{https://drive.google.com}
\end{itemize}

\subsection{Grive2}

Grive2 (fork de grive) permite sincronizar el contenido de la carpeta Google Drive desde linux, plataforma para la que actualmente no existe cliente oficial.

\begin{itemize}
	\item \url{https://github.com/vitalif/grive2}
\end{itemize}

\section{MongoDB}

MongoDB es un sistema gestor de base de datos NoSQL. Ofrece escalabilidad, rendimiento y gran disponibilidad. El motivo de uso de este tipo de SGBD sobre uno de tipo SQL en este trabajo es que MongoDB utiliza documentos JSON para almacenar los registros, el mismo formato en que se obtienen los datos que deseamos almacenar desde la API de Github.

\begin{itemize}
	\item \url{https://www.mongodb.com}
\end{itemize}

\subsection{Mongoose}

Mongoose es un ODM (Object Document Mapper) para MongoDB y JavaScript. Incluye características como conversión de tipos, validación, construcción de consultas, etc.

\begin{itemize}
	\item \url{http://mongoosejs.com/}
\end{itemize}

\subsection{mLab}

La herramienta mLab es un alojamiento de bases de datos MongoDB en la nube, su plan gratuito permite la creación de varias bases de datos con hasta 500 MB de espacio de almacenamiento cada una.

\begin{itemize}
	\item \url{https://mlab.com/}
\end{itemize}

\subsubsection{Alternativas estudiadas}

MongoDB Atlas ofrece bases de datos MongoDB como servicio pero su plan gratuito solo permite la creación de un clúster.

\begin{itemize}
	\item \url{https://www.mongodb.com/cloud/atlas}
\end{itemize}

\section{Herramientas de integración continua}

Las siguientes herramientas se utilizan en el proceso de integración continua.

\subsection{Travis CI}

Travis CI es un sistema de integración continua. Permite automatizar tareas como la construcción, ejecución de pruebas y despliegue de aplicaciones alojadas en Github.

\begin{itemize}
	\item \url{https://travis-ci.org/}
\end{itemize}

\subsection{Gulp}

Gulp es una librería que permite automatizar diversas tareas comunes en el desarrollo de aplicaciones, como por ejemplo la compilación de código fuente. En este trabajo una de las tareas será compilar TypeScript a JavaScript.

\begin{itemize}
	\item \url{http://gulpjs.com/}
\end{itemize}

\subsubsection{Alternativas estudiadas}

Grunt es la alternativa principal a Gulp como método de automatización de tareas.

\begin{itemize}
	\item \url{https://gruntjs.com/}
\end{itemize}

\subsection{Codebeat}

Codebeat es una herramienta para la ejecución de revisiones automáticas sobre repositorios de código. Tras la evaluación ofrece una puntuación denominada GPA que va de 0 (peor) a 4 (mejor). Esta puntuación se calcula teniendo en cuenta aspectos como complejidad, duplicación y seguimiento de buenas prácticas.

La motivación principal de su uso es que, al contrario que las principales herramientas, cuenta con un motor para el lenguaje TypeScript utilizado en este trabajo.

\begin{itemize}
	\item \url{https://codebeat.co}
\end{itemize}

\subsubsection{Alternativas estudiadas}

Se han tenido en cuenta las herramientas SonarQube y Code Climate.
\begin{itemize}
	\item SonarQube: \url{https://www.sonarqube.org/}
	\item Code Climate: \url{https://codeclimate.com/}
\end{itemize}

\subsection{ZenHub}

ZenHub añade funcionalidades que permiten la gestión de proyectos utilizando metodologías ágiles dentro de Github. En este trabajo se utiliza como tablero de tareas y para generar diagramas burndown de cada sprint.

\begin{itemize}
	\item \url{https://www.zenhub.com}
\end{itemize}

\subsection{Mocha + Chai + Sinon}

Mocha es un framework de pruebas unitarias para Node.js y para navegadores que simplifica el desarrollo de pruebas para aplicaciones asíncronas.

Chai es una librería de aserciones con una sintaxis similar al lenguaje natural, facilitando la comprensión de las pruebas. También cuenta con un sistema de plugins que le permite aumentar su funcionalidad, un ejemplo es el plugin chai-http que permite probar aplicaciones que funcionen mediante peticiones http (por ejemplo aplicaciones REST).

Sinon es una librería que funciona con cualquier framework de pruebas, y provee funcionalidades para el uso de spies, stubs y mocks. 

\begin{itemize}
	\item Mocha: \url{https://mochajs.org/}
	\item Chai: \url{http://chaijs.com/}
	\item Sinon: \url{http://sinonjs.org/}
\end{itemize}

\section{Heroku}

Heroku es una herramienta de la familia PaaS (platform as a service). Permite a los desarrolladores construir, ejecutar y operar con aplicaciones completamente en la nube. Soporta diferentes lenguajes como Java, Node.js, Scala, Clojure, Python, PHP o GO.

\begin{itemize}
	\item \url{https://www.heroku.com}
\end{itemize}

\subsection{Alternativas estudiadas}

Como alternativa a Heroku se ha estudiado OpenShift, herramienta creada por Red Hat que hace uso de Docker. Actualmente los nuevos registros están deshabilitados y únicamente permite fases de prueba de un mes de duración.

\begin{itemize}
	\item \url{https://www.openshift.com/}
\end{itemize}


\section{LaTeX}

\LaTeX\ es un sistema de composición de textos orientado a la producción de documentación técnica y científica. Está formado por un conjunto de macros \TeX.

\begin{itemize}
	\item \url{https://www.latex-project.org/}
\end{itemize}

\subsection{Texmaker}

Texmaker es un editor multiplataforma de \LaTeX\ que integra todas las herramientas necesarias para desarrollar este tipo de documentos.

\begin{itemize}
	\item \url{http://www.xm1math.net/texmaker/}
\end{itemize}

\subsubsection{Alternativas estudiadas}

Como editor \LaTeX\ alternativo a Texmaker se valoró el uso de LaTeXila.

\begin{itemize}
	\item \url{https://wiki.gnome.org/Apps/LaTeXila}
\end{itemize}

\section{Skype empresarial}

Como herramienta de comunicación para llevar a cabo las reuniones en cada sprint se ha utilizado Skype empresarial, incluida en el paquete de Office 365 ofrecido por la Universidad de Burgos. Permite programar y realizar videoconferencias con funcionalidades añadidas como compartir pantalla o mensajería instantánea.

\begin{itemize}
	\item \url{https://www.skype.com/es/business/skype-for-business/}
\end{itemize}

\section{Postman}

Postman es una aplicación que permite la prueba y monitorización de API's mediante una interfaz que permite hacer peticiones de diferente tipo (GET, POST, PUT...), modificar las cabeceras, etc.

\begin{itemize}
	\item \url{https://www.getpostman.com/}
\end{itemize}