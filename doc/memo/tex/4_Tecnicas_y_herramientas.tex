\capitulo{4}{Técnicas y herramientas}

En este apartado se definen las diferentes técnicas y herramientas utilizadas a lo largo del desarrollo del trabajo.

\section{Metodologías ágiles}

Frente a las metodologías tradicionales, se ha optado por el uso de metodologías ágiles, las cuales priman \cite{agilealliance:agilemanifesto}:

\begin{itemize}
	\item Individuos e interacciones sobre procesos y herramientas.
	\item Software funcional sobre  documentación.
	\item Colaboración del cliente sobre negociación de contratos.
	\item Respuesta ante cambios sobre seguimiento de un plan.
\end{itemize}

\subsection{Scrum}

Concretamente se ha optado por el uso de Scrum como metodología ágil.

Scrum propone seguir un proceso de desarrollo iterativo e incremental a través de iteraciones denominadas sprints y de revisiones. Cada iteración debe finalizar con la entrega de una parte funcional del producto \cite{scrummanager}.

\section{Git}

Git es un sistema de control de versiones distribuido diseñado para manejar proyectos de cualquier tamaño con velocidad y eficiencia.

\begin{itemize}
	\item \url{https://www.git-scm.com}
\end{itemize}

\section{Github}

Github es una plataforma que permite alojar proyectos en repositorios de código que utilizan el sistema de control de versiones Git. También cuenta con funcionalidades para realizar revisiones de código. En este trabajo se va a utilizar esta herramienta como repositorio de código y como fuente de datos sobre revisiones de código en diferentes repositorios.

\begin{itemize}
	\item \url{https://www.github.com}
\end{itemize}

\subsection{Alternativas estudiadas}

Bitbucket es una alternativa a Github como herramienta de repositorio de código.

\begin{itemize}
	\item \url{https://bitbucket.org}
\end{itemize}

En el ámbito de herramienta de revisión de código se estudiaron Gerrit Code Review y Review Board.

\begin{itemize}
	\item Gerrit Code Review: \url{https://www.gerritcodereview.com/}
	\item Review Board: \url{https://www.reviewboard.org/}
\end{itemize}

\section{Node.js}

Node.js es un entorno de ejecución para JavaScript construido con el motor de JavaScript v8 de Chrome. En este trabajo se utiliza para ejecutar código JavaScript en el lado del servidor.

\begin{itemize}
	\item \url{https://nodejs.org}
\end{itemize}

\subsection{npm}

Node.js cuenta con un npm, un gestor de paquetes con un amplio número de librerías registradas.

\begin{itemize}
	\item \url{https://www.npmjs.com/}
\end{itemize}

\section{TypeScript}

TypeScript es un superconjunto tipado de JavaScript que es compilado a JavaScript plano. Está desarrollado por Microsoft, y su uso puede mejorar la legibilidad y el entendimiento del código con respecto a JavaScript.

\begin{itemize}
	\item \url{https://www.typescriptlang.org/}
\end{itemize}

\section{Visual Studio Code}

Visual Studio Code es un entorno de desarrollo integrado (IDE) multiplataforma desarrollado por Microsoft, cuenta con una herramienta integrada para el uso de Git, así como un elevado número de extensiones que permiten personalizar el editor para las necesidades concretas de cada proyecto.

\begin{itemize}
	\item \url{https://code.visualstudio.com/}
\end{itemize}

\section{Google Drive}

Google Drive es un servicio de almacenamiento en la nube, ofrece 15 GB de almacenamiento gratuito y se utiliza en este trabajo como sistema de copias de seguridad y como medio para mantener el entorno de desarrollo sincronizado en cualquier máquina.

\begin{itemize}
	\item \url{https://drive.google.com}
\end{itemize}

\section{MongoDB}

MongoDB es un sistema gestor de base de datos NoSQL. Ofrece escalabilidad, rendimiento y gran disponibilidad. El motivo de uso de este tipo de SGBD sobre uno de tipo SQL en este trabajo es que MongoDB utiliza documentos JSON para almacenar los registros, el mismo formato en que se obtienen los datos que deseamos almacenar desde la API de Github.

\begin{itemize}
	\item \url{https://www.mongodb.com}
\end{itemize}

\section{Herramientas de integración continua}

Las siguientes herramientas se utilizan en el proceso de integración continua.

\subsection{Travis CI}

Travis CI es un sistema de integración continua. Permite automatizar tareas como la construcción, ejecución de pruebas y despliegue de aplicaciones alojadas en Github.

\begin{itemize}
	\item \url{https://travis-ci.org/}
\end{itemize}

\subsection{Gulp}

TODO

\subsubsection{Alternativas estudiadas}

TODO - Grunt

\subsection{SonarQube o Code Climate}

TODO

\subsection{ZenHub}

ZenHub añade funcionalidades que permiten la gestión de proyectos utilizando metodologías ágiles dentro de Github. En este trabajo se utiliza como tablero de tareas y para generar diagramas burndown de cada sprint.

\begin{itemize}
	\item \url{https://www.zenhub.com}
\end{itemize}

\subsection{Mocha + Chai}

TODO - pruebas unitarias

\section{LaTeX}

LaTeX es un sistema de composición de textos orientado a la producción de documentación técnica y científica. Está formado por un conjunto de macros TeX.

\begin{itemize}
	\item \url{https://www.latex-project.org/}
\end{itemize}

\subsection{Texmaker}

TODO

\subsubsection{Alternativas estudiadas}

TODO - LaTeXila

\section{Skype empresarial}

Como herramienta de comunicación para llevar a cabo las reuniones en cada sprint se ha utilizado Skype empresarial, incluida en el paquete de Office 365 ofrecido por la Universidad de Burgos. Permite programar y realizar videoconferencias con funcionalidades añadidas como compartir pantalla o mensajería instantánea.

\begin{itemize}
	\item \url{https://www.skype.com/es/business/skype-for-business/}
\end{itemize}

\section{Heroku}

TODO - Heroku

\subsection{Alternativas estudiadas}

TODO - Openshift