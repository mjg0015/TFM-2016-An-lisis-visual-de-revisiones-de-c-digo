\apendice{Plan de Proyecto Software}

\section{Introducción}

En este anexo se tratan dos puntos principales:

\begin{itemize}
	\item Planificación temporal del proyecto, donde se han utilizado metodologías ágiles.
	\item Estudio de la viabilidad del proyecto tanto a nivel económico como legal.
\end{itemize}

\section{Planificación temporal}

La planificación temporal del proyecto se caracteriza por el uso de metodologías ágiles, concretamente Scrum. El desarrollo se ha dividido en una serie de iteraciones (o \textit{sprints}).

Se han realizado un total de 14 \textit{sprints}. Cada uno de ellos está enfocado a una serie de objetivos particulares. Cada \textit{sprint} se corresponde con una \textit{milestone} del repositorio en GitHub. A su vez, cada \textit{milestone} está compuesta por una serie de tareas (o \textit{issues}).

Cada una de las iteraciones finaliza con una reunión presencial o mediante videoconferencia donde se revisan y aceptan los cambios. Tras ello, los cambios aceptados son integrados en la rama principal (rama \textit{master}).

El \textit{add-on} ZenHub se ha utilizado para asignar estimaciones temporales a las tareas, las cuales permiten el desarrollo de un diagrama \textit{burndown} por \textit{sprint}. Para las estimaciones de tiempo se utiliza la unidad \textit{story point}, a continuación se muestra una tabla de equivalencias aproximadas:

\tablaSmallSinColores{Equivalencia \textit{story-point} a tiempo.}{l l}{story-point}
{ \textit{Story points} & Tiempo \\}{ 
1 & 30 minutos \\
2 & 1 hora \\
3 & 2 horas \\
4 & 3 horas \\
5 & 6 horas \\
6 & 8 horas \\
8 & 10 horas \\
13 & 16 horas \\
}

A continuación se muestran las iteraciones del proyecto:

\subsection{\textit{Sprint} 1 (10/02/2017 - 24/02/2017)}

Esta iteración del proyecto contiene tareas relacionadas con el aprendizaje de conceptos y herramientas. Sus dos tareas suman una estimación de 20 \textit{story points}.

\url{https://github.com/mjuez/TFM2016_Analisis-Visual-Revisiones-Codigo/milestone/1}

\imagen{sprint1}{Diagrama \textit{burndown} del \textit{sprint} 1.}

\subsection{\textit{Sprint} 2 (24/02/2017 - 10/03/2017)}

Esta iteración del proyecto contiene tareas relacionadas con el desarrollo de la memoria y prototipos previos. Sus 4 tareas suman una estimación de 22 \textit{story points}.

\url{https://github.com/mjuez/TFM2016_Analisis-Visual-Revisiones-Codigo/milestone/2}

\imagen{sprint2}{Diagrama \textit{burndown} del \textit{sprint} 2.}

\subsection{\textit{Sprint} 3 (10/03/2017 - 24/03/2017)}

Esta iteración del proyecto contiene tareas relacionadas con el desarrollo de la memoria, prototipos previos y estudio de tecnologías a utilizar. Sus 5 tareas suman una estimación de 17 \textit{story points}.

\url{https://github.com/mjuez/TFM2016_Analisis-Visual-Revisiones-Codigo/milestone/3}

\imagen{sprint3}{Diagrama \textit{burndown} del \textit{sprint} 3.}

\subsection{\textit{Sprint} 4 (24/03/2017 - 07/04/2017)}

Esta iteración del proyecto contiene tareas relacionadas con el desarrollo de la memoria, prototipos previos y estudio de herramientas de revisión automática. Sus 6 tareas suman una estimación de 18 \textit{story points}.

\url{https://github.com/mjuez/TFM2016_Analisis-Visual-Revisiones-Codigo/milestone/4}

\imagen{sprint4}{Diagrama \textit{burndown} del \textit{sprint} 4.}

\subsection{\textit{Sprint} 5 (07/04/2017 - 21/04/2017)}

Esta iteración del proyecto contiene tareas relacionadas con el estudio de la API de GitHub y el desarrollo de la parte \textit{backend} de la aplicación final. Sus 8 tareas suman una estimación de 32 \textit{story points}.

\url{https://github.com/mjuez/TFM2016_Analisis-Visual-Revisiones-Codigo/milestone/5}

\imagen{sprint5}{Diagrama \textit{burndown} del \textit{sprint} 5.}

\subsection{\textit{Sprint} 6 (21/04/2017 - 28/04/2017)}

Esta iteración del proyecto contiene tareas relacionadas con el estudio de la de las revisiones en GitHub, desarrollo de documentación, desarrollo de la parte \textit{backend} de la aplicación final, y creación de pruebas unitarias. Sus 7 tareas suman una estimación de 24 \textit{story points}.

\url{https://github.com/mjuez/TFM2016_Analisis-Visual-Revisiones-Codigo/milestone/6}

\imagen{sprint6}{Diagrama \textit{burndown} del \textit{sprint} 6.}

\subsection{\textit{Sprint} 7 (28/04/2017 - 12/05/2017)}

Esta iteración del proyecto contiene tareas relacionadas con el desarrollo y finalización de la parte \textit{backend} de la aplicación final. Sus 9 tareas suman una estimación de 34 \textit{story points}.

\url{https://github.com/mjuez/TFM2016_Analisis-Visual-Revisiones-Codigo/milestone/7}

\imagen{sprint7}{Diagrama \textit{burndown} del \textit{sprint} 7.}

\subsection{\textit{Sprint} 8 (12/05/2017 - 19/05/2017)}

Esta iteración del proyecto contiene tareas relacionadas con refactorizaciones de la parte \textit{backend} e inicio de la parte \textit{frontend}. Sus 4 tareas suman una estimación de 20 \textit{story points}.

\url{https://github.com/mjuez/TFM2016_Analisis-Visual-Revisiones-Codigo/milestone/8}

\imagen{sprint8}{Diagrama \textit{burndown} del \textit{sprint} 8.}

\subsection{\textit{Sprint} 9 (19/05/2017 - 26/05/2017)}

Esta iteración del proyecto contiene tareas relacionadas con el desarrollo de la parte \textit{frontend} y alguna mejora en la parte \textit{backend}. Sus 5 tareas suman una estimación de 20 \textit{story points}.

\url{https://github.com/mjuez/TFM2016_Analisis-Visual-Revisiones-Codigo/milestone/9}

\imagen{sprint9}{Diagrama \textit{burndown} del \textit{sprint} 9.}

\subsection{\textit{Sprint} 10 (26/05/2017 - 02/06/2017)}

Esta iteración del proyecto contiene tareas relacionadas con el desarrollo de la parte \textit{frontend}. Sus 6 tareas suman una estimación de 33 \textit{story points}.

\url{https://github.com/mjuez/TFM2016_Analisis-Visual-Revisiones-Codigo/milestone/10}

\imagen{sprint10}{Diagrama \textit{burndown} del \textit{sprint} 10.}

\subsection{\textit{Sprint} 11 (02/06/2017 - 09/06/2017)}

Esta iteración del proyecto contiene tareas relacionadas con la finalización del desarrollo de la parte \textit{frontend} y alguna mejora en la parte \textit{backend}. Sus 8 tareas suman una estimación de 23 \textit{story points}.

\url{https://github.com/mjuez/TFM2016_Analisis-Visual-Revisiones-Codigo/milestone/11}

\imagen{sprint11}{Diagrama \textit{burndown} del \textit{sprint} 11.}

\subsection{\textit{Sprint} 12 (09/06/2017 - 19/06/2017)}

Esta iteración del proyecto contiene tareas relacionadas con la implementación de las últimas características de la aplicación final, documentación de código, y desarrollo de pruebas unitarias. Sus 7 tareas suman una estimación de 34 \textit{story points}.

\url{https://github.com/mjuez/TFM2016_Analisis-Visual-Revisiones-Codigo/milestone/12}

\imagen{sprint12}{Diagrama \textit{burndown} del \textit{sprint} 12.}

\subsection{\textit{Sprint} 13 (19/06/2017 - 26/06/2017)}

Esta iteración del proyecto contiene tareas relacionadas con la finalización del desarrollo de la memoria del proyecto. Sus 11 tareas suman una estimación de 44 \textit{story points}.

\url{https://github.com/mjuez/TFM2016_Analisis-Visual-Revisiones-Codigo/milestone/13}

\imagen{sprint13}{Diagrama \textit{burndown} del \textit{sprint} 13.}

\subsection{\textit{Sprint} 14 (26/06/2017 - 03/07/2017)}

Ésta es la última iteración del proyecto. Sus tareas están relacionadas con el desarrollo de los anexos (documentación técnica).

\url{https://github.com/mjuez/TFM2016_Analisis-Visual-Revisiones-Codigo/milestone/14}

En el momento de realización de este documento, la iteración aún no había terminado, y por tanto no existía el diagrama \textit{burndown}.


\section{Estudio de viabilidad}

A continuación se desarrolla el estudio de viabilidad del proyecto, tanto a nivel económico como a nivel legal.

\subsection{Análisis coste-beneficio}

El estudio de viabilidad económica consiste en el análisis de costes y beneficios del proyecto.

\subsubsection{Costes}

Los costes del proyecto se han desglosado en los siguientes tipos: \newline

\textbf{Costes de personal:}

El proyecto ha sido realizado por un desarrollador empleado a media jornada durante 3 meses, y a jornada completa durante 2 meses:

\tablaSmallConPie{Costes de personal a media jornada.}{l l}{costes-personal-mj}
{ \textbf{Descripción} & \textbf{Coste} \\}{ 
Salario mensual neto & 550,00 \euro{} \\
Retención IRPF (19\%) & 204,50 \euro{} \\
Seguridad Social (29,9\%) & 321,82 \euro{} \\
Salario mensual bruto & 1.076,32 \euro{} \\
}{
\textbf{Total 3 meses} & 3.228,96 \euro{} \\
}

\tablaSmallConPie{Costes de personal a jornada completa.}{l l}{costes-personal-jc}
{ \textbf{Descripción} & \textbf{Coste} \\}{ 
Salario mensual neto & 1.100,00 \euro{} \\
Retención IRPF (19\%) & 409,00 \euro{} \\
Seguridad Social (29,9\%) & 643,64 \euro{} \\
Salario mensual bruto & 2.152,64 \euro{} \\
}{
\textbf{Total 2 meses} & 4.305,28 \euro{} \\
}

Los costes de Seguridad Social (29,9\%) se han calculado tal como indica el Ministerio de Empleo y Seguridad Social \cite{seguridadsocial}:
\begin{itemize}
	\item 23,60\% por contingencias comunes.
	\item 5,50\% por desempleo de tipo general.
	\item 0,20\% para el Fondo de Garantía Salarial.
	\item 0,60\% por formación profesional.
\end{itemize}

\textbf{Costes de \textit{hardware}:}

Para el desarrollo del proyecto se ha utilizado un ordenador portátil con un coste de 990 \euro{} y una vida útil de 5 años. El coste amortizado es de 82,5 \euro.

Para calcular el coste de amortización se ha utilizado la siguiente fórmula:

$$ Coste Amortizado = \frac{Coste Hardware}{Vida Util} * Tiempo Utilizado $$ \newline

\textbf{Costes de \textit{software}:}

Todo el \textit{software} utilizado durante el desarrollo del proyecto es gratuito o tiene planes gratuitos. \newline

\textbf{Otros costes:}

A continuación se detalla el resto de costes del proyecto:

\tablaSmallConPie{Otros costes.}{l l}{otros-costes}
{ \textbf{Descripción} & \textbf{Coste} \\}{ 
Internet & 165,25 \euro{} \\
Documentación (memoria impresa + \textit{CD's}) & 50,00 \euro{} \\
}{
\textbf{Total} & 215,25 \euro{} \\
}

\textbf{Costes totales:}

La suma de todos los costes del proyecto es la siguiente:

\tablaSmallConPie{Costes totales.}{l l}{costes-totales}
{ \textbf{Descripción} & \textbf{Coste} \\}{ 
Personal & 7.534,24 \euro{} \\
\textit{Hardware} & 82,50 \euro{} \\
\textit{Software} & 0,00 \euro{} \\
Otros & 215,25 \euro{} \\
}{
\textbf{Total} & 7.831,99 \euro{} \\
}

\subsubsection{Beneficios}

Este proyecto se ha desarrollado para ser utilizado en entornos de investigación universitaria. No se espera obtener beneficio económico del mismo.

\subsection{Viabilidad legal}

El estudio de la viabilidad legal se ha realizado en términos relativos a las licencias utilizadas.

\subsubsection{Licencia del \textit{software} desarrollado}

Como se ha indicado anteriormente, nuestro proyecto tiene fines académicos y de investigación. Por ello no queremos restringir su uso, distribución o modificación.

Para la elección de la licencia de la aplicación desarrollada se han tenido en cuenta las licencias de las dependencias externas utilizadas.

\tablaSmallSinColores{Diferentes licencias utilizadas por nuestras dependencias.}{l l}{licencias}
{ \textbf{Licencia} & \textbf{Tipo} \\}{ 
WTFPL & Copyright \\
MIT & Copyright \\
ISC & Copyright \\
Apache 2.0 & Copyright \\
BSD-3-Clause & Copyright \\
}

La licencia MIT \cite{mit} se ajusta a nuestras necesidades. Es compatible con todas las anteriores, y además es poco restrictiva. Permite, entre otras cosas, el uso, copia, modificación o distribución del \textit{software} con la única condición de incluir nuestro aviso de copyright y licencia en toda copia o derivado.

La licencia GPLv3 también era compatible con las licencias de nuestras dependencias, pero al ser de tipo copyleft, es algo más restrictiva que la MIT.

\subsubsection{Licencia de la documentación}

En lo relativo a la documentación (memoria y anexos), se ha utilizado una licencia de tipo Creative Commons 4.0 \cite{creativecommons} (CC BY 4.0), que permite la copia y redistribución, así como su modificación para cualquier propósito.