\apendice{Documentación de usuario}

\section{Introducción}

Este anexo contiene la documentación necesaria para que el usuario conozca y aprenda el funcionamiento de la aplicación.

El usuario siempre accederá a la última versión de la aplicación debido a que ésta se encuentra desplegada en un servidor remoto y no precisa de instalación local. Se sigue un proceso de desarrollo incremental, lo que supone que la funcionalidad de la aplicación está en constante evolución. Por ello la documentación de usuario debe ser actualizada periódicamente.

Para que el usuario siempre tenga a su disposición la última versión de la documentación de usuario, ésta se encuentra accesible desde la \textit{Wiki} del repositorio.

\section{Requisitos}

La aplicación desarrollada se despliega en la nube, lo que quiere decir que no necesita instalación.

Para el uso de la API REST, el usuario deberá tener instalado cualquier software que permita realizar peticiones y recibir respuestas HTTP. Cualquier navegador web permite este tipo de operaciones, aunque también existen aplicaciones desarrolladas específicamente para trabajar con APIs como por ejemplo Postman.

Para el uso del cliente web, será necesario utilizar la última versión de uno de los tres navegadores principales:

\begin{itemize}
	\item{Microsoft Edge}
	\item{Mozilla Firefox}
	\item{Google Chrome}
\end{itemize}

También es imprescindible que el navegador tenga el uso de JavaScript habilitado. La aplicación debería funcionar correctamente en otros navegadores (por ejemplo Safari), pero solo se ha probado en los tres indicados anteriormente.

\section{Manual del usuario}

Los manuales de usuario se pueden encontrar en la \textit{Wiki} del repositorio, en el siguiente enlace:

\codigo{\url{https://github.com/mjuez/TFM2016_Analisis-Visual-Revisiones-Codigo/wiki}}

\imagen{wikiGitHub}{Fragmento de la \textit{Wiki} en GitHub.}

