\capitulo{3}{Conceptos teóricos}

En este apartado se exponen una serie de conceptos teóricos relacionados con las revisiones de código. Se define cada concepto así como su relación con el presente trabajo.

\section{Revisión de código}
La revisión de código es un proceso de ingeniería a través del cual se realiza una inspección del código fuente por otros desarrolladores diferentes al autor del mismo. Resulta útil para reducir los defectos de código así como mejorar la calidad del software \cite{28121}.

Frente a las revisiones de código altamente estructuradas propuestas por Fagan \cite{5388086}, hoy en día se están adoptando metodologías más livianas con el fin de solventar las ineficiencias de las inspecciones de código. Las denominadas \emph{Modern Code Reviews} son informales, hacen uso de herramientas, y se utilizan regularmente.

Mediante el uso de estas nuevas metodologías, las revisiones de código ofrecen un mayor número de beneficios a los equipos de desarrollo como transferencia de conocimientos, visión de equipo, o mejores soluciones a los problemas \cite{Bacchelli:2013:EOC:2486788.2486882}.

Las revisiones de código son el elemento principal de este trabajo, donde se van a obtener datos de las mismas realizadas en diversos repositorios de software.

\subsection{Inspección de software}
La inspección de software, también conocida como inspección de Fagan.

\imagen{fagan_inspection}{Operaciones de la inspección de software}

\subsection{Revisión por pares}

\subsection{Revisión automática de código}

\section{Revisor de código}
\subsection{Tipos de revisores}
\section{Control de versiones}
\subsection{Git}
\subsection{Repositorio}
\subsection{Cambio (diff)}
\subsection{Rama (branch)}
\subsection{Integración (merge)}
\subsection{Revision}
\subsection{Version (blob)}
\subsection{Árbol (tree)}
\subsection{Cambio (commit)}
\subsection{Etiqueta (tag)}
\subsection{Pull request}
\section{Integración contínua}