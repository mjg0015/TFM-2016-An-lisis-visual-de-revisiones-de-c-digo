\capitulo{3}{Conceptos teóricos}

En este apartado se exponen una serie de conceptos teóricos relacionados con las revisiones de código. Se define cada concepto así como su relación con el presente trabajo.

\section{Revisión de código}
La revisión de código es un proceso de ingeniería a través del cual se realiza una inspección del código fuente por otros desarrolladores diferentes al autor del mismo. Resulta útil para reducir los defectos de código así como mejorar la calidad del software \cite{28121}.

Frente a las revisiones de código altamente estructuradas propuestas por Fagan \cite{5388086}, hoy en día se están adoptando metodologías más livianas con el fin de solventar las ineficiencias de las inspecciones de código. Las denominadas \emph{Modern Code Reviews} son informales, hacen uso de herramientas, y se utilizan regularmente.

Mediante el uso de estas nuevas metodologías, las revisiones de código ofrecen un mayor número de beneficios a los equipos de desarrollo como transferencia de conocimientos, visión de equipo, o mejores soluciones a los problemas \cite{Bacchelli:2013:EOC:2486788.2486882}.

Las revisiones de código son el elemento principal de este trabajo, donde se van a obtener datos de las mismas realizadas en diversos repositorios de software.

\subsection{Inspección de software}
La inspección de software, también conocida como inspección de Fagan, consiste en la búsqueda de defectos en documentos de desarrollo (código, especificaciones, diseño, etc.) por parte de personas con diversos roles mediante un proceso estructurado y bien definido, con el fin de disminuir el número de errores y aumentar la calidad del producto final \cite{5388086}.

Las fases de dicho proceso son las siguientes \cite{Fagan:1986:ASI:10677.2412491}:

\begin{description}
	\item[Planificación] En esta fase, los materiales que van a ser inspeccionados deben coincidir con los criterios de entrada de la inspección. Tiene lugar la elección de los participantes de la inspección. También se realiza la organización de las reuniones (hora y lugar).
	\item[Información general] En esta fase se comunica a los participantes cuales son los resultados esperados. También se realiza la asignación de roles.
	\item[Preparación] En esta fase tiene lugar el estudio del material por parte de los participantes y su preparación para cumplir sus roles.
	\item[Reunión de inspección] Esta es la fase donde se lleva a cabo la búsqueda de defectos.
	\item[Repetición del trabajo] El autor debe solucionar los defectos detectados, tras ello todo el proceso vuelve a comenzar desde la fase de planificación.
	\item[Seguimiento] El moderador o equipo completo de inspección debe verificar que todos los cambios son correctos y no introducen nuevos defectos.
\end{description}

\imagen{faganInspection}{Operaciones de la inspección de software}

Además, durante el proceso de inspección de software intervienen los siguientes roles:

\begin{description}
	\item[Autor] Responsable del desarrollo del documento a inspeccionar (código).
	\item[Lector] Responsable de leer el documento como si fuese a desarrollarlo él mismo.
	\item[Revisor] Responsable de examinar e inspeccionar el documento con el fin de identificar posibles defectos.
	\item[Moderador] Responsable de coordinar la reunión de inspección.
\end{description}

\section{Revisor de código}

El revisor de código tiene como responsabilidad principal examinar el código y detectar posibles defectos, ofreciendo comentarios y sugerencias de cambio al autor del código que permitan la mejora de la calidad del mismo.

Los aportes del revisor de código también pueden proporcionar nuevos conocimientos al autor del código revisado, mejorando así sus habilidades lo cual se puede traducir en una mejora progresiva de la calidad del software que desarrolle.

Se pueden distinguir dos tipos de revisores de código:

\begin{description}
	\item[Revisor experto] Es una persona, idealmente con experiencia y amplios conocimientos (aunque no es estrictamente necesario).
	\item[Robot revisor] Son herramientas que permiten la revisión automática de código mediante la comprobación del código fuente para garantizar que cumpla un conjunto de reglas predefinidas, así como detectar errores \cite{wiki:001}.
\end{description}

Actualmente es común que las revisiones cuenten con ambos tipos de revisores, en primer lugar se realiza una revisión automática, y los resultados son utilizados por el revisor experto para ofrecer una mejor revisión con un mayor nivel de detalle.

El revisor de código es una figura importante de este trabajo. Nos interesa extraer los comentarios que registra en cada cambio puesto que pueden resultar útiles para obtener métricas que podrían permitir evaluar la calidad de los comentarios o incluso del propio revisor.

\section{Sistema de control de versiones}

Un sistema de control de versiones se encarga de registrar los diferentes cambios de un fichero o un conjunto de ficheros a lo largo del tiempo, permitiendo así recuperar versiones específicas en cualquier momento \cite{Chacon:2014:PG:2695634}.

Gracias a este tipo de sistemas es posible revertir los cambios realizados en un fichero o incluso en un proyecto completo a un estado anterior, comparar los cambios, comprobar la autoría de los mismos, etc.

Los sistemas de control de versiones están en constante evolución, actualmente herramientas como Github están empezando a implementar funcionalidades para facilitar  las revisiones de código. Por tanto, en este trabajo vamos a utilizar este tipo de sistemas como fuente de datos a extraer.

\subsection{Repositorio}

El repositorio es la parte central de un sistema de control de versiones, es el lugar donde se almacenan los datos del sistema, normalmente estructurados en forma de árbol de ficheros \cite{Pilato:2008:VCS:1435405}.

La característica principal de un repositorio es que mantiene todas las versiones de cada fichero, permitiendo al cliente recuperar estados previos de los mismos.

\subsection{Cambio (diff)}

Un cambio es una modificación concreta de un fichero alojado en en un repositorio de un control de versiones \cite{wiki:002}.

Resulta útil para comparar dos versiones del mismo archivo.

\subsection{Rama (branch)}

Una rama es un apuntador a una confirmación. Por cada confirmación realizada, la rama avanza hasta la última confirmación. Por defecto existe una rama principal \cite{Chacon:2014:PG:2695634}. 

Crear otras ramas sirve para abrir nuevas líneas de desarrollo, por ejemplo una rama de pruebas, evitando así desordenar la principal.

\subsection{Integración (merge)}

Se entiende por integración a la unión de los cambios de dos ramas en una. Existen dos patrones de uso de esta operación \cite{sbf5:git3}:

\begin{itemize}
	\item Añadir los cambios realizados en una rama donde se están implementando nuevas características a la rama principal.
	\item Añadir los cambios de la rama principal en una rama donde se están implementando nuevas características con el fin de mantener la rama actualizada con los últimos parches y características. Esta práctica permite reducir el riesgo de conflictos\footnote{Existe un conflicto cuando al integrar dos ramas, un mismo fichero ha sido modificado en ambas y por ello es necesario decidir que cambios deben mantenerse en la versión final.} al unir la rama de desarrollo con la principal.
\end{itemize}

\subsection{Confirmación (commit)}

Una confirmación (o commit) supone el almacenamiento en el repositorio de una nueva versión de los ficheros que contiene, en otras palabras, sirve para guardar los cambios \cite{Chacon:2014:PG:2695634}.

Cada confirmación lleva asociados uno o varios ficheros modificados, un identificador denominado revisión y un mensaje donde se describen los cambios realizados.

\subsection{Etiqueta (tag)}

La finalidad del etiquetado es dar nombre a alguna versión para que pueda ser localizada facilmente en el futuro, es decir, permite identificar de forma sencilla revisiones importantes del proyecto \cite{wiki:002}.

\subsection{Pull request}

Las "Pull Requests" son una característica concreta de la herramienta Github. Sirven para mantener conversaciones sobre los cambios donde se pueden exponer ideas, asignar tareas, tratar detalles y hacer revisiones de código \cite{github:features}.

\section{Integración contínua}

La integración continua es una práctica de desarrollo software donde los miembros de un equipo integran su trabajo frecuentemente, normalmente como mínimo una integración diaria \cite{fowler2006continuous}.

Cada integración se verifica mediante la automatización de ciertos procesos como construcción, pruebas y despliegue. Mediante este tipo de verificación se busca detectar errores de integración tan rápido como sea posible.

Para el desarrollo de este trabajo se utilizarán técnicas de integración contínua.
