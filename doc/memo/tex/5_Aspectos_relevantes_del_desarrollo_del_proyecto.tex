\capitulo{5}{Aspectos relevantes del desarrollo del proyecto}

En este apartado se recogen los aspectos más interesantes del desarrollo del proyecto.

\section{Ciclo de vida del proyecto}

Este proyecto tiene una duración total de 4 meses y 22 días, con inicio el 10 de febrero de 2017 y fin el 3 de julio de 2017.

Como se ha utilizado la metodología ágil Scrum, el proyecto está dividido en sprints (o iteraciones), concretamente 14. Inicialmente las iteraciones tenían una duración de 2 semanas, y a partir del sexto sprint la periodicidad pasó a ser semanal.

Se pueden distinguir tres fases principales:

\begin{description}
	\item[Prototipado] En esta fase se deciden, estudian y prueban diversas tecnologías como paso previo a la decisión de cuales utilizar en el desarrollo del producto final.
	\item[Desarrollo de la aplicación] Se trata de la fase más importante, abarca todo el desarrollo de la aplicación que a su vez se podría dividir en dos: desarrollo de la parte backend (API REST) y desarrollo de la parte frontend (cliente SPA).
	\item[Desarrollo de la memoria] Esta fase consiste en escribir toda la documentación relativa al proyecto (memoria y anexos).
\end{description}

En el siguiente gráfico se muestra el porcentaje de sprints dedicados a cada una de las fases. Es un gráfico orientativo puesto que no hay un punto exacto en el cual termina la fase de prototipado y empieza la de desarrollo de la aplicación. Asimismo, en la fase de prototipado, una vez decididas las herramientas y tecnologías hubo avances en la fase de desarrollo de memoria.

\imagen{graficoTipoIteraciones}{Porcentaje de sprints dedicados a cada fase.}

\subsection{Gestión de sprints mediante GitHub y ZenHub}

Cada sprint está formado por una serie de tareas concretas.

Para la gestión y planificación de dichas tareas se han utilizado las issue de GitHub junto con el plug-in de ZenHub que permite, entre otras cosas, incluir estimaciones de tiempo.

\imagen{issueGitHub}{Una tarea de GitHub con ZenHub habilitado.}

\section{Estableciendo la base}

TODO

\subsection{Entornos: producción y desarrollo}

TODO

\subsection{Travis y la integración continua}

TODO

\subsection{Variables de entorno frente a fichero de configuración}

TODO

\subsection{TypeScript (JavaScript tipado)}

TODO

\subsection{Documentos JSON y MongoDB}

TODO -> aqui mlab?

\subsection{Arquitectura cliente-servidor}

TODO

\subsubsection{API REST independiente del cliente}

TODO

\subsubsection{Cliente SPA}

TODO

\subsection{Revisiones automáticas con Codebeat}

TODO

\section{Obteniendo datos de GitHub}

TODO

\subsection{¿Qué datos son necesarios?}

TODO

\subsection{El problema de las 5000 peticiones/hora}

TODO

\subsection{Algoritmo de gestión de tareas}

TODO

\section{Visualizando datos mediante gráficas}

TODO

\section{Otros aspectos}

TODO

\subsection{Gestión de memoria: paginando resultados}

TODO

\subsection{Testing y la inyección de dependencias}

TODO