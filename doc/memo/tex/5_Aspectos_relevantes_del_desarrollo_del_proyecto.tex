\capitulo{5}{Aspectos relevantes del desarrollo del proyecto}

En este apartado se recogen los aspectos más interesantes del desarrollo del proyecto.

\section{Ciclo de vida del proyecto}

Este proyecto tiene una duración total de 4 meses y 22 días, con inicio el 10 de febrero de 2017 y fin el 3 de julio de 2017.

Como se ha utilizado la metodología ágil Scrum, el proyecto está dividido en sprints (o iteraciones), concretamente 14. Inicialmente las iteraciones tenían una duración de 2 semanas, y a partir del sexto sprint la periodicidad pasó a ser semanal.

Se pueden distinguir tres fases principales:

\begin{description}
	\item[Prototipado] En esta fase se deciden, estudian y prueban diversas tecnologías como paso previo a la decisión de cuales utilizar en el desarrollo del producto final.
	\item[Desarrollo de la aplicación] Se trata de la fase más importante, abarca todo el desarrollo de la aplicación que a su vez se podría dividir en dos: desarrollo de la parte backend (API REST) y desarrollo de la parte frontend (cliente SPA).
	\item[Desarrollo de la memoria] Esta fase consiste en escribir toda la documentación relativa al proyecto (memoria y anexos).
\end{description}

En el siguiente gráfico se muestra el porcentaje de sprints dedicados a cada una de las fases. Es un gráfico orientativo puesto que no hay un punto exacto en el cual termina la fase de prototipado y empieza la de desarrollo de la aplicación. Asimismo, en la fase de prototipado, una vez decididas las herramientas y tecnologías hubo avances en la fase de desarrollo de memoria.

\imagen{graficoTipoIteraciones}{Porcentaje de sprints dedicados a cada fase.}

\subsection{Gestión de sprints mediante GitHub y ZenHub}

Cada sprint está formado por una serie de tareas concretas.

Para la gestión y planificación de dichas tareas se han utilizado las issue de GitHub junto con el plug-in de ZenHub que permite, entre otras cosas, incluir estimaciones de tiempo.

\imagen{issueGitHub}{Una tarea de GitHub con ZenHub habilitado.}

Se utilizan tres ramas:

\begin{description}
	\item[master] Rama principal, contiene los últimos cambios estables.
	\item[dev] Rama de desarrollo, se va actualizando a medida que se realizan tareas. Es una rama inestable, la aplicación puede no tener el comportamiento esperado.
	\item[memo] Rama de memoria, contiene los cambios no definitivos de la documentación del proyecto.
\end{description}

\imagen{ramasGitHub}{Ejemplo de uso de rama dev en GitHub.}

El ciclo de vida de un sprint en GitHub consiste en:

\begin{enumerate}
	\item Apertura de issues (tareas) a realizar durante la iteración.
	\item Selección de rama dev o memo, dependiendo del tipo de tarea que se está llevando a cabo.
	\item Desarrollo de las diferentes tareas.
	\item Revisión y comprobación de que los cambios satisfacen los requisitos.
	\item Apertura de pull request para ejecutar los checks de integración continua y revisión automática.
	\item Cierre de pull request y fusión de los cambios en la rama master.
\end{enumerate}

%\imagen{checksPullRequest}{Checks de integración continua y revisión automática.}

\section{Estableciendo la base}

Antes de empezar con el desarrollo software del proyecto se tomaron ciertas decisiones que marcaron el resto del proceso.

\subsection{Anvireco como aplicación distribuida}

La herramienta desarrollada tiene una arquitectura cliente-servidor en la cual el servidor es completamente independiente del cliente.

Pese a que en este proyecto únicamente hay un tipo de cliente, en el futuro podrían existir otros, por ejemplo un consumidor de datos para realizar tareas de minería con ellos. Por ello es importante la independencia cliente-servidor.

\imagen{aplicacionDistribuida}{Anvireco como aplicación distribuida.}

En el servidor se optó por implementar una API REST con la cual podrá comunicarse cualquier cliente que conozca el protocolo HTTP.

El cliente de visualización de datos es una aplicación web de tipo SPA por los beneficios que ofrece el uso de AJAX en cuanto a optimización de peticiones y disminución de los refrescos de página, ofreciendo así una mejor experiencia de usuario.

\imagen{tradicionalVersusSPA}{Página web tradicional vs Aplicación SPA \cite{msdn:spa}}

\subsection{Entornos: producción y desarrollo}

En la sección del ciclo de vida del proyecto se ha indicado el uso de diferentes ramas Git durante el desarrollo. Esto se traduce en la utilización de dos entornos diferentes e independientes donde se despliega la aplicación:

\begin{description}
	\item[Entorno de producción] En este entorno se despliegan los cambios de la rama master, es decir, la aplicación en un estado estable.
	\item[Entorno de desarrollo] En este entorno se despliegan los cambios de la rama dev, es decir, la aplicación puede tener un comportamiento inestable.
\end{description}

\imagen{entornos}{Entornos de producción y desarrollo del proyecto.}


\subsection{Travis y la integración continua}

TODO

\subsection{Variables de entorno frente a fichero de configuración}

TODO

\subsection{Revisiones automáticas con Codebeat}

TODO

\section{Obteniendo datos de GitHub}

TODO

\subsection{¿Qué datos son necesarios?}

TODO

\subsection{El problema de las 5000 peticiones/hora}

TODO

\subsection{Algoritmo de gestión de tareas}

TODO

\section{Visualizando datos mediante gráficas}

TODO

\section{Otros aspectos}

TODO

\subsection{Gestión de memoria: paginando resultados}

TODO

\subsection{Testing y la inyección de dependencias}

TODO